\section{Datastructure Flattening}

In order to run the code efficiently on the GPGPU we had to start with 2 steps:
\begin{enumerate*}
	\item Flatten all 2D structures (vectors in vectors) to flat versions.
    \item Replace all \texttt{std::vector} occurrences with C arrays.
\end{enumerate*}

The first step is done to simplify the datastructures we pass back and forth
and minimize the number of pointer traversals needed to look up values. The
second step is needed because CUDA does not support the STL and thus we can't
have usage of vectors in our device code. 

%%%%%%%%%%%%%%%%%%%%%%%%%%%%%%%%%%%%%%%%%%%%%%%%%%%%%%%%%%%%%%%%%%%%%%%%%%%%%%%%
\subsection{Flattenening 2D vectors}

In order to flatten the 2D vectors in the \texttt{PrivGlobs} structure each
2D vector (matrix) was replaced with a single vector where the length would then
be extended to contain all the elements of the matrix. In order to not loose
the information about the dimensions which could previusly be obtained by
calling the \texttt{size()} function on the root vector or a row two variables
where added to hold the information. Figure \ref{code:privglobsflatten} shows
the translation for the \texttt{myResult} part of the datastructure. Technically
the number of rows is still contained in \texttt{myResult.size()}, but since we
will replace the vectors with arrays in the next subsection, this counter was
added right away.

\begin{figure}[H]
\begin{minipage}{.45\textwidth}
\begin{lstlisting}
vector<vector<REAL> > myResult; // [numX][numY]
//...
this->myResult.resize(numX);
for(unsigned i=0;i<numX;++i) {
    this->myResult[i].resize(numY);
}
\end{lstlisting}
\end{minipage}\hfill
\begin{minipage}{.45\textwidth}
\begin{lstlisting}
vector<REAL> myResult; // [numX][numY]
unsigned myResultRows;
unsigned myResultCols;
//...
this->myResult.resize(numX*numY);
this->myResultRows = numX;
this->myResultCols = numY;
\end{lstlisting}
\end{minipage}
\caption{Left: Sample of the original \texttt{PrivGlobs} structure that shows a
2D vector. Right: The flattened structure and new additional information.}
\label{code:privglobsflatten}
\end{figure}

In order for the code that utilized these 2D vectors to still work we had to
implement a method that would allow to to calculate the offset into the one
dimensional vector that corresponded to a coordinate in the 2D vector. The code
shown in Figure \ref{code:idx2d} shows a function that when given a coordinate
into a matrix and the number of columns/width of the matrix will calculate the
linear offset into our flat vector version.

\begin{figure}[H]
    \begin{lstlisting}
unsigned idx2d(unsigned row, unsigned col, unsigned width) {
    return row*width+col;
}
    \end{lstlisting}
    \caption{The 2D offset calculation function.}
    \label{code:idx2d}
\end{figure}

An example of how code was altered from using the 2D vectors to using our flat
vectors can be seen in Figure \ref{code:2dto1dcoord}. The surrounding loops are
included to give context to the $i$ and $j$ variables, the figure also shows how
the new column and row information is retrieved as opposed to using the
\texttt{size()} function.

\begin{figure}[H]
    \begin{minipage}{.45\textwidth}
        \begin{lstlisting}
for(unsigned i=0 ;
    i < globs.myX.size() ;
    ++i) {
    for(unsigned j=0 ;
        j<globs.myY.size() ;
        ++j) 
        globs.myResult[i][j] = payoff[i];
}
        \end{lstlisting}
    \end{minipage}\hfill
    \begin{minipage}{.45\textwidth}
        \begin{lstlisting}
for(unsigned i=0 ;
    i < globs.myXsize ;
    ++i) {
    for(unsigned j=0 ;
        j < globs.myYsize ;
        ++j) 
        globs.myResult[idx2d(i,j,globs.myResultCols)] = payoff[i];
}
        \end{lstlisting}
    \end{minipage}
    \caption{Left: The original way to access the 2D structure. Right: The how
    to access the flattened version. The example is from the
    \texttt{setPayoff} function.}
    \label{code:2dto1dcoord}
\end{figure}

Lastly, any functions that have 2D vectors in the signature, such as
\texttt{iniOperator}, will have to have their signature changed. Using
\texttt{initOperator} as an example, the change would be from \texttt{
void initOperator(  const vector<REAL> \&x, vector<vector<REAL> > \&Dxx );} to
\texttt{void initOperator(  const vector<REAL> \&x, unsgined xSize, vector<REAL>
\&Dxx, unsigned DxxCols );}.

%%%%%%%%%%%%%%%%%%%%%%%%%%%%%%%%%%%%%%%%%%%%%%%%%%%%%%%%%%%%%%%%%%%%%%%%%%%%%%%%
\subsection{Replacing vectors with arrays}

Replacing the \texttt{vector} datastrcutures with arrays was rather simple since
much of the preparation such as shifting from 2D to 1D coordinates and creating
the needed extra data such as row and column numbers were done in the previus
step. The first part of this change is changing \texttt{PrivGlobs} as shown in
Figure \ref{code:privglobsv2a}.

\begin{figure}[H]
    \begin{minipage}{.45\textwidth}
        \begin{lstlisting}
vector<REAL> myResult; // [numX][numY]
unsigned myResultRows;
unsigned myResultCols;
//...
this->myResult.resize(numX*numY);
this->myResultRows = numX;
this->myResultCols = numY;
        \end{lstlisting}
    \end{minipage}\hfill
    \begin{minipage}{.45\textwidth}
        \begin{lstlisting}
REAL* myResult; // [numX][numY]
unsigned myResultRows;
unsigned myResultCols;
//...
this->myResult = (REAL*) malloc(sizeof(REAL)*numX*numY);
this->myResultRows = numX;
this->myResultCols = numY;
        \end{lstlisting}
    \end{minipage}
    \caption{Left: The flat vector implementation. Right: The array
    implementation that allows us to transfer everything to the GPGPU device.}
    \label{code:privglobsv2a}
\end{figure}

The second part is to change the function signatures around the code to use
\texttt{REAL*} instead of \texttt{vector<REAL>}, if we again use
\texttt{initOperator} as an example we change from \texttt{void initOperator(
const vector<REAL> \&x, unsgined xSize, vector<REAL> \&Dxx, unsigned DxxCols );}
to \texttt{void initOperator(  const REAL* \&x, unsgined xSize, REAL* \&Dxx,
unsigned DxxCols );}.