\documentclass[a4paper,11pt]{article}
\usepackage{a4wide}
\usepackage[utf8x]{inputenc}
\usepackage{ucs}
\usepackage[T1]{fontenc}
\linespread{1.2}
\usepackage{amsmath,amssymb,amsthm,amsfonts,ulem}
\usepackage{courier}
% \usepackage{fourier}
\usepackage{color}
% \usepackage{clrscode3e}
\usepackage{multicol}
%\usepackage{pdflscape}
\setcounter{secnumdepth}{2}
\setcounter{tocdepth}{3}

\usepackage{hyperref}
\usepackage{listings}
\usepackage{subcaption}
\usepackage{float}
\usepackage{mdwlist}
\usepackage{wrapfig}
\usepackage{caption}
\usepackage{todonotes}
\usepackage{ulem}

\usepackage{mathtools}
\DeclarePairedDelimiter{\ceil}{\lceil}{\rceil}

% usage: \graphicc{width}{file}{caption}{label}
\newcommand{\graphicc}[4]{\begin{figure}[H] \centering
            \includegraphics[width={#1\textwidth}, keepaspectratio=true]{{#2}}
            \caption{{#3}} \label{#4} \end{figure}}

% usage: \codefig{label}{file}{firstline}{lastline}{description}
\newcommand{\codefig}[5]
{
\begin{figure}[H]
    \lstinputlisting[firstnumber=#3,firstline=#3,lastline=#4]{#2}
    \caption{#5 (#2)}
    \label{code:#1}
\end{figure}
}



\definecolor{comment}{rgb}      {0.38, 0.62, 0.38}
\definecolor{keyword}{rgb}      {0.10, 0.10, 0.81}
\definecolor{identifier}{rgb}   {0.00, 0.00, 0.00}
\definecolor{string}{rgb}       {0.50, 0.50, 0.50}

\lstset
{
    language=c++,
    % general settings
    numbers=left,
    frame=single,
    basicstyle=\footnotesize\ttfamily,
    tabsize=4,
    breaklines=true,
    showstringspaces=false,
    % syntax highlighting
    commentstyle=\color{comment},
    keywordstyle=\color{keyword},
    identifierstyle=\color{identifier},
    stringstyle=\color{string},
}

\title{\textbf{Final Project Report\\ Programming Massively Parallel Hardware 2015}}
\author
{
    Martin Jørgensen \\
    University of Copenhagen \\
    Department of Computer Science \\
    {\tt tzk173@alumni.ku.dk}
    \and
    Henrik Bendt \\
    University of Copenhagen \\
    Department of Computer Science \\
    {\tt gwk553@alumni.ku.dk}
}
\date{\today}

\begin{document}

\maketitle

\tableofcontents
\pagebreak


%Object:
% This project targets first-and-foremost efficient parallelization of the small dataset. This dataset corresponds to values OUTER=16, NUM_X = 32, NUM_Y = 256, and requires that all parallelism is exploited in order to efficiently utilize the GPGPU, i.e., the loops in the tridag function must also be parallelized. The latter requires the computation of several segmented scans (interleaved with maps) in the innermost dimension. However the segment size is either 32 or 256, which means that they can be performed at CUDA-block level, hence efficient. (Meaning, if the block size is chosen 256 then the elements of a segment will never cross two blocks, and in particular there is a multiple of segments that would fit exactly the size of the block.) You will receive ample help on how to reqrite tridag into segmented scans and how to parallelize the scans.



%Notes
% vi skal i rapporten argumentere for hvad vi kan og ikke kan parallelisere
% og så kommer det til at være meget privatization og loop expansion
% og så interchange det yderste, som ma ruler dem ud (som jeg lige forstod det)
% og så er der self noget loop distribution

% og han råder os til at lave CUDA som det sidste, efter alle loop-fixes
% vi starter i noget C++ kode, med vectors, fordi de simplificere koden - men skal ende i ren C fordi vi bruger CUDA
% så vi bør vente med CUDA-transformation til aller sidst.


\section{Introduction}
In this report we explain the ideas behind parallelizing the given system. This was done by flattening the system via array expansion and privatization of non-array variables, which are later transformed into inlined scalar variables. To increase the degree of parallelism, loop distribution was applied. To keep memory access coalesced, loops where interchanged where possible and/or matrices where transposed. 

The system has one inherently sequential loop, in between parallel loops, which was distributed to be the outer loop via array expansions (making the inner arrays a dimension higher) and loop distribution.

Unfortunately we were unable to complete the programming part. We misused a lot of time on the convertion from the flat sequential program to CUDA kernels, whereby we created a PrivGlob structure completely on CUDA (to omit having to copy arrays to and from the kernels). This however made debugging hell, and after tens of hours debugging without valid results we had to revert back to the flat sequential solution and create kernels one at a time, with the PrivGlobs placed completely in hos memory. We should have had realized this sooner, but alas we where stubborn thinking a valid solutions was right around the next logical error.

Due to this, we ran out of time before parallelizing the rollback part, which performance wise is also the most essential part. This also means that we have not optimized the given tridag kernel. However, this report should still give the ideas behind parallelizing the system. % Introduction... Duh.
\section{Datastructure Flattening}

In order to run the code efficiently on the GPGPU we had to start with 2 steps:
\begin{enumerate*}
	\item Flatten all 2D structures (vectors in vectors) to flat versions.
    \item Replace all \texttt{std::vector} occurrences with C arrays.
\end{enumerate*}

The first step is done to simplify the datastructures we pass back and forth
and minimize the number of pointer traversals needed to look up values. The
second step is needed because CUDA does not support the STL and thus we can't
have usage of vectors in our device code. 

%%%%%%%%%%%%%%%%%%%%%%%%%%%%%%%%%%%%%%%%%%%%%%%%%%%%%%%%%%%%%%%%%%%%%%%%%%%%%%%%
\subsection{Flattenening 2D vectors}

In order to flatten the 2D vectors in the \texttt{PrivGlobs} structure each
2D vector (matrix) was replaced with a single vector where the length would then
be extended to contain all the elements of the matrix. In order to not loose
the information about the dimensions which could previusly be obtained by
calling the \texttt{size()} function on the root vector or a row two variables
where added to hold the information. Figure \ref{code:privglobsflatten} shows
the translation for the \texttt{myResult} part of the datastructure. Technically
the number of rows is still contained in \texttt{myResult.size()}, but since we
will replace the vectors with arrays in the next subsection, this counter was
added right away.

\begin{figure}[H]
\begin{minipage}{.45\textwidth}
\begin{lstlisting}
vector<vector<REAL> > myResult; // [numX][numY]
//...
this->myResult.resize(numX);
for(unsigned i=0;i<numX;++i) {
    this->myResult[i].resize(numY);
}
\end{lstlisting}
\end{minipage}\hfill
\begin{minipage}{.45\textwidth}
\begin{lstlisting}
vector<REAL> myResult; // [numX][numY]
unsigned myResultRows;
unsigned myResultCols;
//...
this->myResult.resize(numX*numY);
this->myResultRows = numX;
this->myResultCols = numY;
\end{lstlisting}
\end{minipage}
\caption{Left: Sample of the original \texttt{PrivGlobs} structure that shows a
2D vector. Right: The flattened structure and new additional information.}
\label{code:privglobsflatten}
\end{figure}

In order for the code that utilized these 2D vectors to still work we had to
implement a method that would allow to to calculate the offset into the one
dimensional vector that corresponded to a coordinate in the 2D vector. The code
shown in Figure \ref{code:idx2d} shows a function that when given a coordinate
into a matrix and the number of columns/width of the matrix will calculate the
linear offset into our flat vector version.

\begin{figure}[H]
    \begin{lstlisting}
unsigned idx2d(unsigned row, unsigned col, unsigned width) {
    return row*width+col;
}
    \end{lstlisting}
    \caption{The 2D offset calculation function.}
    \label{code:idx2d}
\end{figure}

An example of how code was altered from using the 2D vectors to using our flat
vectors can be seen in Figure \ref{code:2dto1dcoord}. The surrounding loops are
included to give context to the $i$ and $j$ variables, the figure also shows how
the new column and row information is retrieved as opposed to using the
\texttt{size()} function.

\begin{figure}[H]
    \begin{minipage}{.45\textwidth}
        \begin{lstlisting}
for(unsigned i=0 ;
    i < globs.myX.size() ;
    ++i) {
    for(unsigned j=0 ;
        j<globs.myY.size() ;
        ++j) 
        globs.myResult[i][j] = payoff[i];
}
        \end{lstlisting}
    \end{minipage}\hfill
    \begin{minipage}{.45\textwidth}
        \begin{lstlisting}
for(unsigned i=0 ;
    i < globs.myXsize ;
    ++i) {
    for(unsigned j=0 ;
        j < globs.myYsize ;
        ++j) 
        globs.myResult[idx2d(i,j,globs.myResultCols)] = payoff[i];
}
        \end{lstlisting}
    \end{minipage}
    \caption{Left: The original way to access the 2D structure. Right: The how
    to access the flattened version. The example is from the
    \texttt{setPayoff} function.}
    \label{code:2dto1dcoord}
\end{figure}

Lastly, any functions that have 2D vectors in the signature, such as
\texttt{iniOperator}, will have to have their signature changed. Using
\texttt{initOperator} as an example, the change would be from \texttt{
void initOperator(  const vector<REAL> \&x, vector<vector<REAL> > \&Dxx );} to
\texttt{void initOperator(  const vector<REAL> \&x, unsgined xSize, vector<REAL>
\&Dxx, unsigned DxxCols );}.

%%%%%%%%%%%%%%%%%%%%%%%%%%%%%%%%%%%%%%%%%%%%%%%%%%%%%%%%%%%%%%%%%%%%%%%%%%%%%%%%
\subsection{Replacing vectors with arrays}

Replacing the \texttt{vector} datastrcutures with arrays was rather simple since
much of the preparation such as shifting from 2D to 1D coordinates and creating
the needed extra data such as row and column numbers were done in the previus
step. The first part of this change is changing \texttt{PrivGlobs} as shown in
Figure \ref{code:privglobsv2a}.

\begin{figure}[H]
    \begin{minipage}{.45\textwidth}
        \begin{lstlisting}
vector<REAL> myResult; // [numX][numY]
unsigned myResultRows;
unsigned myResultCols;
//...
this->myResult.resize(numX*numY);
this->myResultRows = numX;
this->myResultCols = numY;
        \end{lstlisting}
    \end{minipage}\hfill
    \begin{minipage}{.45\textwidth}
        \begin{lstlisting}
REAL* myResult; // [numX][numY]
unsigned myResultRows;
unsigned myResultCols;
//...
this->myResult = (REAL*) malloc(sizeof(REAL)*numX*numY);
this->myResultRows = numX;
this->myResultCols = numY;
        \end{lstlisting}
    \end{minipage}
    \caption{Left: The flat vector implementation. Right: The array
    implementation that allows us to transfer everything to the GPGPU device.}
    \label{code:privglobsv2a}
\end{figure}

The second part is to change the function signatures around the code to use
\texttt{REAL*} instead of \texttt{vector<REAL>}, if we again use
\texttt{initOperator} as an example we change from \texttt{void initOperator(
const vector<REAL> \&x, unsgined xSize, vector<REAL> \&Dxx, unsigned DxxCols );}
to \texttt{void initOperator(  const REAL* \&x, unsgined xSize, REAL* \&Dxx,
unsigned DxxCols );}.   % A word about how/why we flattened all the datastructures to simple arrays.
\section{CUDA Preperation \& OpenMP}

In this step we will cover the transformations we applied to the code on order
to make it easier to parallelize in the end. 

%%%%%%%%%%%%%%%%%%%%%%%%%%%%%%%%%%%%%%%%%%%%%%%%%%%%%%%%%%%%%%%%%%%%%%%%%%%%%%%%
\subsection{OpenMP}
The first task was to use OpenMP to parllelize the outermost loop in
\texttt{run\_OrigCPU} by adding an OpenMP pragma to it. The resulting loop can
be seen in Figure \ref{code:openmp1}.

\begin{figure}[H]
    \begin{lstlisting}
#pragma omp parallel for default(shared) schedule(static) if(outer>8)
for( unsigned i = 0; i < outer; ++ i ) {
    REAL strike = 0.001*i;
    PrivGlobs    globs(numX, numY, numT);
    res[i] = value( globs, s0, strike, t,
                    alpha, nu,    beta,
                    numX,  numY,  numT );
}
    \end{lstlisting}
    \caption{The outer loop parallelized using OpenMP.}
    \label{code:openmp1}
\end{figure}

Originally the declerations of \textit{strike} and \textit{globs} that we see in
Figure \ref{code:openmp1} was placed outside the loop, but in order to
parallelize the loop they had to be privatized. If they had stayed outside the
loop each different value of strike and globs would have been mapped to the same
memory location and the different threads would be writing to, and reading from
the same location instead of individual locations. This is needed since the
values for each iteration of the loop are not the same. The above
parallelization is safe because no loop iteration (thread) reads or writes to
any shared variables, and the rest of the code (everything inside the
\texttt{value} function) is executed sequentially.

%%%%%%%%%%%%%%%%%%%%%%%%%%%%%%%%%%%%%%%%%%%%%%%%%%%%%%%%%%%%%%%%%%%%%%%%%%%%%%%%
\subsection{Array Expansion}

One of the transformatinos we applied to the program was array expansion as
explained in \cite[Slide 11]{projectslide}. Array expansion essentially adds an
extra dimension to whatever data you are working with, but works better than
privatization when parallelizing for a GPU since dynamically allocating memory
in a CUDA kernel is not possible. If we use the loop from Figure
\ref{code:openmp1} as an example, we would allocate an array of
\texttt{PrivGlobs} before starting the loop, and then access each individual
\texttt{PrivGlobs} for each iteration of the loop, the result would look like
the code shown in Figure \ref{code:arrayexp1}. Array expansion is used to
increase the degree of parallelization since we can now run as many threads as
the loop have iterations.

\begin{figure}[H]
    \begin{lstlisting}
PrivGlobs *globs = (PrivGlobs*) malloc(outer*sizeof(struct PrivGlobs));

for(int i = 0 ; i < outer ; i++) { //par
    globs[i] = PrivGlobs(numX,numY,numT);
}

for( unsigned i = 0; i < outer; ++ i ) { //par
        REAL strike = 0.001*i;
        res[i] = value( globs[i], s0, strike, t,
                    alpha, nu,    beta,
                    numX,  numY,  numT );
}
    \end{lstlisting}
    \caption{The outer loop modified to use array expansion instead of
    privatization.}
    \label{code:arrayexp1}
\end{figure}

The code in Figure \ref{code:arrayexp1} is using OpenMP, so the array expansion
is not strictly needed since it can dynamically allocate memory, but the step is
done to prepare the code for being translated to CUDA. This transformation is
applied other places in the code, but this outer loop is the clearest example.

%%%%%%%%%%%%%%%%%%%%%%%%%%%%%%%%%%%%%%%%%%%%%%%%%%%%%%%%%%%%%%%%%%%%%%%%%%%%%%%%
\subsection{Loop Distribution}

Another transformation we applied was loop distribution as described in
\cite[Slide 13]{projectslide}. This optimization is applied when you have an
outer loop that can be parallelized, with other parallelizable inner loops, that
have some sequential code between them. The outer loop again provided a good
example of this when we look into the \texttt{value} function as well. Figure
\ref{code:predistvalue} shows the \texttt{value} function, it contains some
function calls and sequential loop.

\begin{figure}[H]
    \begin{lstlisting}
REAL   value(   PrivGlobs    globs,
                const REAL s0,
                const REAL strike,
                const REAL t,
                const REAL alpha,
                const REAL nu,
                const REAL beta,
                const unsigned int numX,
                const unsigned int numY,
                const unsigned int numT
) {
    initGrid(s0,alpha,nu,t, numX, numY, numT, globs);
    initOperator(globs.myX,globs.myDxx);
    initOperator(globs.myY,globs.myDyy);

    setPayoff(strike, globs);
    for(int i = globs.myTimeline.size()-2;i>=0;--i) {//seq
        updateParams(i,alpha,beta,nu,globs);
        rollback(i, globs);
    }

    return globs.myResult[globs.myXindex][globs.myYindex];
}
    \end{lstlisting}
    \caption{The \texttt{value} function as it was handed out.}
    \label{code:predistvalue}
\end{figure}

If we pull the code from the \texttt{value} function out into the outer loop
directly, it will look like Figure \ref{code:arrayexp2}, since the functions
\texttt{initGrid} and \texttt{initOperator} as well as \texttt{setPayoff} were
safe to parallelize we reorganized the loops to the form in Figure
\ref{code:arrayexp3}.

\begin{figure}[H]
    \begin{lstlisting}
PrivGlobs *globs = (PrivGlobs*) malloc(outer*sizeof(struct PrivGlobs));

for(int i = 0 ; i < outer ; i++) { //par
    globs[i] = PrivGlobs(numX,numY,numT);
}

for( unsigned i = 0; i < outer; ++ i ) { //par
        REAL strike = 0.001*i;
        initGrid(s0,alpha,nu,t, numX, numY, numT, globs);
        initOperator(globs.myX,globs.myDxx);
        initOperator(globs.myY,globs.myDyy);

        setPayoff(strike, globs);

        for(int j = globs.myTimeline.size()-2;j>=0;--j) //seq
        {
            updateParams(j,alpha,beta,nu,globs);
            rollback(j, globs);
        }

        res[i] = globs[i].myResult[globs.myXindex][globs.myYindex];
}
    \end{lstlisting}
    \caption{The outer loop after value is expanded into it.}
    \label{code:arrayexp2}
\end{figure}

Figure \ref{code:arrayexp3} shows the loops after they were disitributed, this
transformation allowed us to now run all the init operations in parallel, which
increased the degree of parallelization. Furthermore it allowed us to
interchange the sequential inner loop and the parallel outerloop which again
allowed us top increase the degree of parallelism.

\begin{figure}[H]
    \begin{lstlisting}
PrivGlobs *globs = (PrivGlobs*) malloc(outer*sizeof(struct PrivGlobs));

for(int i = 0 ; i < outer ; i++) { //par
    globs[i] = PrivGlobs(numX,numY,numT);
}

for( unsigned i = 0; i < outer; ++ i ) { //par
        initGrid(s0,alpha,nu,t, numX, numY, numT, globs[i]);
        initOperator(globs[i].myX, globs[i].myXsize, globs[i].myDxx, globs[i].myDxxCols);
        initOperator(globs[i].myY, globs[i].myYsize, globs[i].myDyy, globs[i].myDyyCols);
        setPayoff(0.001*i, globs[i]);
}

// sequential loop distributed.
for(int i = numT-2;i>=0;--i){ //seq
    #pragma omp parallel for default(shared) schedule(static) if(outer>8)
    for( unsigned j = 0; j < outer; ++ j ) { //par
        updateParams(i,alpha,beta,nu,globs[j]);
        rollback(i, globs[j]);
    }
}
    \end{lstlisting}
    \caption{The same code as in \ref{code:arrayexp2} but after the loops are
    distributed.}
    \label{code:arrayexp3}
\end{figure}

%%%%%%%%%%%%%%%%%%%%%%%%%%%%%%%%%%%%%%%%%%%%%%%%%%%%%%%%%%%%%%%%%%%%%%%%%%%%%%%%
\subsection{Memory Coalescing}     % How we prepared the code for parallelization using loop interchange, array expansion and so forth.
\section{CUDA Rewrite}
The globs run on host memory. We tried to make it all CUDA (i.e. all arrays of the globs lying in device memory), but we could not get it to work. A problem with making it all CUDA is that it is very memory heavy, which might not be prefferable, but as this project was on optimizing only for the small test sample with focus on speed performance and less focus on memory efficiency, we argued that this could be more effecient. This should hovewer have been one of the last things to focus on, after having a running CUDA implementation, thus also making it possible to benchmark versus a glob lying on host memory.

The whole initialization of the globs is translated to CUDA kernels and works. This is placed in the \texttt{InitKernels.cu.h} file. The \texttt{updateParms} part of the sequential loop (before \texttt{rollback}) also runs on CUDA. Both these are parallelized to the outer most parallel loop (that is, until the main sequential loop) with degree of parallelism $outer$.  All loops are distributed across kernels of their specific parallelism degree.

\texttt{InitKernels.cu.h} contains several kernels, all with a maximum parallelization degree based on their loops, that is, between $outer\cdot numT$ for \texttt{initGritTimeline} to $outer\cdot numX \cdot numY$ for \texttt{setPayoff}. \texttt{updateParams} is parallel in $outer\cdot numX \cdot numY$ dimensions. 

The kernels for \texttt{rollback} are placed in file \texttt{CoreKernels.cu.h}, but unfortunately contains some bugs. We tried to include them one by one, running the rest sequentially, without success due and time did not allow us to further debug it. But all kernels are created and some of them might even work.

Had time allowed it, we would have also liked to improve the \texttt{TRIDAG\_SOLVER} used as tridag kernel in \texttt{rollback}. Here, shared memory is limited and could be optimized e.g. maybe reusing \textit{mat\_sh} for \textit{lin\_sh}, as \textit{lin\_sh} is only used after the last use of \textit{mat\_sh} (so it is pretty straight forward). Also, it would be a point of optimization if $uu$ could be put in shared/tiled memory as it is firstly written to and then accessed in the previous index (which must be read from memory).         % How we transformed the code into CUDA and what choices where made and how.
\section{Benchmarks}
Due to the system not being fully converted to CUDA, specifically the essential and heavy rollback part, the CUDA implementation runs much slower than its OpenMP counter parts. This is due to rollback is not parallelized in any way and also a lot of copying memory back and forth to between the host and device memory.

\begin{table}[H]
    \centering
    \begin{tabular}{r|r|r|r}
                             &  Small Dataset & Medium Dataset &    Large Dataset \\ \hline
        Sequential Handout   & $2242717\mu s$ & $4925513\mu s$ & $244242016\mu s$ \\
        OpenMP Handout       &  $204197\mu s$ &  $323792\mu s$ &  $10555286\mu s$ \\
        OpenMP + Preperation & $1593800\mu s$ & $1802730\mu s$ &  $24765850\mu s$ \\
        OpenMP + Flattening  & $1411000\mu s$ & $1826789\mu s$ &  $20222463\mu s$ \\
        CUDA                 & $2744045\mu s$ & $4329038\mu s$ & $129250226\mu s$ \\
    \end{tabular}
    \caption{Benchmark results for running the different iteration of the code
        on each of the datasets. All times are an average of 5 runs measured in
        microseconds.}
    \label{tab:benchmarks}
\end{table}



%%%%%%%% BENCHMARK DATA %%%%%%%%


%%%%% SEQUENTIAL HANDOUT
%%% SMALL  % 2105254+2151681+2142690+2639859+2174101 WRITTEN
%%% MEDIUM % 4443326+4430587+4444841+5651892+5656919 WRITTEN
%%% LARGE  % 190664573+197486649+192104500+192104500+217292287+231557573 WRITTEN

%%%%% OPENMP HANDOUT
%%%%% 00_HandoutImpl
%%% SMALL  % 190505+183517+230618+222014+194331 WRITTEN
%%% MEDIUM % 287590+318034+365399+340666+307275 WRITTEN
%%% LARGE  % 10343945+10395789+11147729+10462171+10426797 WRITTEN

%%%%% OPENMP PREPPED
%%%%% 02_HandoutExpanded
%%% SMALL  % 1609894+1572326+1629429+1544463+1612890 WRITTEN
%%% MEDIUM % 1974435+1759373+1690479+1744996+1844369 WRITTEN
%%% LARGE  % 24965604+24570208+24983224+24829709+24480505 WRITTEN

%%%%% OPENMP FLATTENED
%%%%% 04_FlatArrayImpl
%%% SMALL  % 1347727+1456623+1410288+1409993+1430370 WRITTEN
%%% MEDIUM % 1958594+1817759+1856793+1703780+1797023 WRITTEN
%%% LARGE  % 20254772+20261447+20310689+20166920+20118490 WRITTEN

%%%%% FULL CUDA
%%% SMALL  %
%%% MEDIUM %
%%% LARGE  %   % Results when running the different data sizes for each version using different datasets.
\section{Conclusion}

We did not get a full CUDA-implementation to work but only a partly CUDA-translated version with the most important part non-parallelized, namely the \texttt{rollback} loop.

We have in this report attempted to explain both abstractly and concretely via code examples the work process and goals of parallelizing the given system.

This includes flattening the implementation, loop distributing, creating coalesced memory accesses and loop interchanging along with transposing of matrices and moving sequential loops to the outer most, as was the goal of the project.

The benchmarks are incomplete, as the CUDA-version is incomplete. This makes the benchmarks perform considerably worse than the CPU parallel versions. We do not expect this to be the case, had the CUDA-version been fully implemented, and especially not if we had time to optimize it also, e.g. the \texttt{TRIDAG\_SOLVER}.


% In this report we explain the ideas behind parallelizing the given system. This was done by first flattening the vectors/arrays of the system. To increase the degree of parallelism, array expansion and privatization of simple non-array variables, which are later transformed into inlined scalar variables, was applied. To further increase the degree of parallelism over the whole system, loop distribution was applied throughout the system from the main loop. To keep memory access coalesced, loops where interchanged where possible and/or matrices where transposed. 

% The system has one inherently sequential loop, in between parallel loops, which was distributed to be the outer loop via array expansions (making the inner arrays a dimension higher) and loop distribution.

% Unfortunately we were unable to complete the programming part. We misused a lot of time on the convertion from the flat sequential program to CUDA kernels, whereby we created a PrivGlob structure completely on CUDA (to omit having to copy arrays to and from the kernels). This however made debugging hell, and after tens of hours debugging without valid results we had to revert back to the flat sequential solution and create kernels one at a time, with the PrivGlobs placed completely in hos memory. We should have had realized this sooner, but alas we where stubborn thinking a valid solutions was right around the next logical error.

% Due to this, we ran out of time before parallelizing the rollback part, which performance wise is also the most essential part. This also means that we have not optimized the given tridag kernel. However, this report should still give the ideas behind parallelizing the system.   % The final, epic conclusion.


%%\appendix
%%\bibliographystyle{abbrv}
%%\bibliography{citations}
%%\input{appendix1}

\end{document}